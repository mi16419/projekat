 % !TEX encoding = UTF-8 Unicode

\documentclass[a4paper]{report}

\usepackage[T2A]{fontenc} % enable Cyrillic fonts
\usepackage[utf8x,utf8]{inputenc} % make weird characters work
\usepackage[serbian]{babel}
%\usepackage[english,serbianc]{babel}
\usepackage{amssymb}

\usepackage{color}
\usepackage{url}
\usepackage[unicode]{hyperref}
\hypersetup{colorlinks,citecolor=green,filecolor=green,linkcolor=blue,urlcolor=blue}

\newcommand{\odgovor}[1]{\textcolor{blue}{#1}}

\begin{document}

\title{Dopunite naslov svoga rada\\ \small{Dopunite autore rada}}

\maketitle

\tableofcontents


\chapter{Recenzent \odgovor{--- ocena: 4} }


\section{O čemu rad govori?}
% Напишете један кратак пасус у којим ћете својим речима препричати суштину рада (и тиме показати да сте рад пажљиво прочитали и разумели). Обим од 200 до 400 карактера.

Rad govori o nastanku, osobinama i upotrebi programskog jezika Swift. U okviru Swift-a moguće je korišćenje koda iz programskih jezika Objective-C, C ili C++, kao i programiranje u više programskih paradigmi. Budući da je u pitanju proizvod kompanije Apple, Swift je pogodan za programiranje aplikacija za macOS i iOS uređaje.

\section{Krupne primedbe i sugestije}
% Напишете своја запажања и конструктивне идеје шта у раду недостаје и шта би требало да се промени-измени-дода-одузме да би рад био квалитетнији.

U sekciji 3 se, u jednom delu, opisuju osnovna namena i svrha; ti delovi više pripadaju uvodu i/ili zaključku.

\odgovor{Poglavlje 3 je posvećeno osnovnoj nameni, svrsi i mogućnostima programskog jezika. U Uvodu ima reči o nameni, ali se u ovom delu malo detaljnije bavimo time.}


U sekciji 4 se govori o okruženjima Sublime Text i Atom; ali, budući da su u pitanju klasični editori teksta sa mogućnosti proširivanja uz pomoć \textit{plugin} sistema, nisu previše vezani za sam programski jezik Swift.

\odgovor{Ispravljeno, dodato je i opisano razvojno okruženje Cocoa Touch, ali su Sublime Text i Atom, zbog svog značaja u Swift programiranju, zadržani.}

\section{Sitne primedbe}
% Напишете своја запажања на тему штампарских-стилских-језичких грешки

U prvoj rečanici sažetka, rečeno je da je Swift "izvanredan" jezik. Takvi epiteti ne bi trebalo da stoje u tekstovima naučnog karaktera. Takođe, reč "performanse" je pogrešno napisana.

\odgovor{Ispravljeno u \textit{"je opšte primenljiv programski jezik "} i \textit{"performanse"}.}


Rečenica u uvodu "Dizajniran je da radi u Apple radnim okruženjima, Cocoa i Cocoa Touch i postojećeg Objective-C koda pisanog za Apple proizvode" nema previše smisla (tačnije, deo koji pominje Objecive-C nema smisla). Zapeta u datoj rečenici je napotrebna.

\odgovor{Obrisana zapeta i deo rečenice koji nije bio smislen.}

U sekciji 2 stoji rečenica "Apple je izbacio...". Mišljenja sam da bi se reč "izbacio" mogla zameniti rečju "objavio". U podsekciji 2.1 se nalazi slika razvojnog stabla koja je malo zbunjujuća - dodavanje strelica bi olakšalo razumevanje dotične slike.

\odgovor{Ispravljeno u \textit{"Apple je zvanično objavio"}. Programski jezici su poređani hronološki, s leva u desno, takođe ispod svakog programskog jezika je napisana godina nastanka, pa smo mišljenja da se na osnovu toga može zaključiti koji su programski jezici na koje uticali.}

U sekciji 3 se pominje mogućnost korišćenja Swift-a na "Linux operativnim sistemima i Raspberry Pi", ali, budući da je Raspberry Pi računar, a ne platforma, pominjanje istog se čini redudantnim.

\odgovor{Ispravljeno u \textit{"je moguće izvršavati i na Linux operativnom sistemu"}. }

Slika 2 u okviru sekcije 4 prikazuje zaradu programera - te zarade bi trebalo sortirati.

\odgovor{Ispravljeno, zarade su sada sortirane.}

U Sekciji 6, operativni sistemi razvijeni od strane Apple-a se zovu \textit{macOS}. Takođe, rečenica u nastavku "...Xcode uključuje izdanje Swift-a koje podržava Apple" nije smislena; logičnije bi bilo "koja podržava macOS sisteme".

\odgovor{Ispravljeno, naziv operativnog sistema je promenjen, rečenica je preformulisana \textit{"Ukoliko se koristi macOS, dovoljno je da se preuzme i instalira Xcode razvojno okruženje, jer Xcode uključuje izdanje Swift-a"} .}

Slika 4 stoji u sred sekcije "Swift na Linux-u", iako je na slici prikazano okruženje za Windows.

\odgovor{Slika se sada nalazi na pravom mestu.}

U okviru kodova, po svemu sudeći, nije korišćen monospace font, tako da je preglednost istih otežana. Takođe, kod nekih kodova postoji prva linija koja je prazna.

\odgovor{Obrisana je svuda prva prazna linija.}

U okviru literature se, u okviru nekih referenci, nalaze nepotrebni tabulatori.

\odgovor{Literatura je ispravljena tako da se sada ne nalaze nepotrebni tabulatori.}

\section{Provera sadržajnosti i forme seminarskog rada}
% Oдговорите на следећа питања --- уз сваки одговор дати и образложење

\begin{enumerate}
\item Da li rad dobro odgovara na zadatu temu?

\odgovor{\textbf{Da.} Rad je, i pored sitnih grešaka, uspeva da omogući čitaocu da se, na adekvatan način, upozna sa programskim jezikom Swift.}

\item Da li je nešto važno propušteno?

\odgovor{\textbf{Nije.} Najvažnije stvari su rečene: mogućnosti Swifta kao programskog jezika, kao i njegove specifičnosti su predočene na odgovarajuć način.}

\item Da li ima suštinskih grešaka i propusta?

\odgovor{\textbf{Ne.} Rad ima nekih krupnijih grešaka, ali to nisu greške koje se mogu okarakterisati kao "suštinske".}

\item Da li je naslov rada dobro izabran?

\odgovor{\textbf{Nije.} Smatram da bi naslov trebalo da bude malo kreativniji.}

\odgovor{Naslov je sada promenjen u \textbf{Razvoj i primena programskog jezika SWIFT}.}

\item Da li sažetak sadrži prave podatke o radu?

\odgovor{\textbf{Da.} Stavljanje informacije da je Swift napravljen od strane Apple-a je dobar način da čitaoca privuče daljem čitanju, budući da se radi o prestižnoj korporaciji.}

\item Da li je rad lak-težak za čitanje?

\odgovor{\textbf{Lak.} Mislim da informatički obrazovani čitaoci neće imati bilo kakav problem sa praćenjem rada.}

\item Da li je za razumevanje teksta potrebno predznanje i u kolikoj meri?

\odgovor{\textbf{Da.} Rad je najverovatnije i namenjen ljudima upoznatim sa osnovnim konceptima programiranja. Osim toga, rad nema specijalne prohteve za razumevanje istog.}

\item Da li je u radu navedena odgovarajuća literatura?

\odgovor{\textbf{Da.} Korišćene su informacije sa oficijelnog Swift sajta ili iz knjiga renomiranih izdavača.}

\item Da li su u radu reference korektno navedene?

\odgovor{\textbf{Ne.} Naišao sam na konfliktne informacije prilikom rada izvesnog Ivice Milovanovića sa Računarskog Fakulteta. Na \href{\underline{https://www.raf.edu.rs/en/component/k2/item/5768-milovanovic-ivica}}{zvaničnom sajtu RAF-a} piše da je u pitanju master rad, a ne doktorska teza, kao što je navedeno. Ostale reference su sasvim u redu.}


\odgovor{Referenca je sada ispravljena.}


\item Da li je struktura rada adekvatna?

\odgovor{\textbf{Da.} Rad sadrži sve neophodne sekcije, koje su navedene u logičnom redosledu.}

\item Da li rad sadrži sve elemente propisane uslovom seminarskog rada (slike, tabele, broj strana...)?

\odgovor{\textbf{Ne.} Rad treba da sadrži referencu na makar jedan naučni žurnal. Osim toga, rad ima 11 strana, dve tabele i više slika, što zadovoljava postavljene zahteve.}

\odgovor{Dodat je naučni rad.}

\item Da li su slike i tabele funkcionalne i adekvatne?

\odgovor{\textbf{Da.} One sasvim adekvatno prikazuju zamisli autora.}

\end{enumerate}

\section{Ocenite sebe}
% Napišite koliko ste upućeni u oblast koju recenzirate: 
% a) ekspert u datoj oblasti
% b) veoma upućeni u oblast
% c) srednje upućeni
\odgovor{d) malo upućeni}
% e) skoro neupućeni
% f) potpuno neupućeni
% Obrazložite svoju odluku


\chapter{Recenzent \odgovor{--- ocena: 3} }


\section{O čemu rad govori?}
% Напишете један кратак пасус у којим ћете својим речима препричати суштину рада (и тиме показати да сте рад пажљиво прочитали и разумели). Обим од 200 до 400 карактера.
	Swift, napravljen od stranje kompanije Apple, počijne kao mali projekat 2010. dok 2013. postaje glavni fokus Apple Developer Tools.
	Nastaje pod uticajem drugih jezika, od kojih je najznačajniji Objective-C, Rust, Haskell, Ruby i Python. Swift preuzima neke od karakteristka drugih programskih jezika.
	Primarno je namenjen za razvijanje iOS aplikacija, ali ima i primene kod modernih serverskih aplikacija. Postaje sve popularniji kod izrade IOT aplikacija. Neke od mogućnosti
	koje pruža Swift jesu automatsko utvrđivanje tipova, generički tipovi i pseudoklase. Najznačajnija osobina veoma je jasan i bezbedan. Zbog posedovanja takvih osobina on je spadao
	u najplaćeniji programski jezik u SAD 2016. godine.	Podržava objektno-orijentisanu, imperativnu i funkcionalnu paradigmu. Najpopularnije okruženje za rad sa Swift-om je Xcode koji je napravila kompanija 
	Apple. Takođe neka od popularnijih okruženja su Sublime i Atom Text editori. Za instalaciju na Windows sistemu koristi se razvojno okruženje, dok se pod linuxom instalira sam kompajler.
	Swift je izuzetno kvalitetan industriski program. Napravljen je tako da maksimalno izvuče od modernog hardvera.


\section{Krupne primedbe i sugestije}
% Напишете своја запажања и конструктивне идеје шта у раду недостаје и шта би требало да се промени-измени-дода-одузме да би рад био квалитетнији.
\begin{itemize}
    \item Kod instalacije za Linux, možda je bolje da se koristi stable verzija kompajlera a ne snapshot.
    
    \odgovor{Ispravljeno je, uputstvo za instalaciju sada vodi do stable verzije kompajlera.}
    
	\item Da se za primer instalacije koristi tcolorbox umesto listings (npr. \\begin{tcolorbox}[colback=green!5!white,colframe=green!5!white,fontupper=\\ttfamily] )
	
    \odgovor{Koristimo listings kroz ceo rad, u delu instalacije on je podešen na odgovarajući način. Zadržaćemo ovu formu. }
    
	\item Mislim da treba da se promeni instalacija kod Windows OS-a. Koristi se razvojno okruženje dok možda bi bilo bolje da se skine sam kompiler (Novija verzija WIN 10 ima WSL (Windows Subsystem for Linux) pa se 
	instalacija pod win vrši potpuno ekvivalentno kao kod Linux sistema).
	
	\odgovor{Naša instalacija odgovara svim verzijama Windowsa.}

	\item Kod primera kodova mislim da bi bilo dobro i da ima izlaz koda koji je napisan.
	
	\odgovor{Ispravljeno, dodat je izlaz za kodove.}

	\item Kod literature, 6 i 7, javlja se mnogo razmaka.
	
	\odgovor{Literatura je ispravljena tako da se sada ne nalaze nepotrebni tabulatori.}
	
\end{itemize}

\section{Sitne primedbe}
% Напишете своја запажања на тему штампарских-стилских-језичких грешки
\odgovor{Nisam zapazio štamparske-stilske-jezičke greške}

\section{Provera sadržajnosti i forme seminarskog rada}
% Oдговорите на следећа питања --- уз сваки одговор дати и образложење

\begin{enumerate}
\item Da li rad dobro odgovara na zadatu temu?\\
\odgovor{ Da, rad pokriva sve tražene teme.}
\item Da li je nešto važno propušteno?\\ 
\odgovor {Ne, rad pokriva sve važne aspekte Swift-a.}
\item Da li ima suštinskih grešaka i propusta?\\
\odgovor {Nisam naišao suštinske greške i propuste.}
\item Da li je naslov rada dobro izabran?\\
\odgovor {Izabran naslov odgovara radu.}
\item Da li sažetak sadrži prave podatke o radu?\\ 
\odgovor {Sažetak sadrži sve važne koncepte koji se javljaju u radu.}
\item Da li je rad lak-težak za čitanje?\\
\odgovor {Rad je lak za čitanje.}
\item Da li je za razumevanje teksta potrebno predznanje i u kolikoj meri?\\ 
\odgovor {Za razumevanje je potrebno osnovno znaje o programiranju.}
\item Da li je u radu navedena odgovarajuća literatura?\\ 
\odgovor {Jeste. Literatura obuhvata rad, knjigu, i tehničku stranicu swift jezika.}
\item Da li su u radu reference korektno navedene?\\
\odgovor {Jesu.}
\item Da li je struktura rada adekvatna?\\ 
\odgovor {Struktura rada je adekvatna. Počinje sa istorijatom, potom govori o samom jeziku i njegovim specifičnostima i završava se sa radnim okruženjima. Zaključak je adekvatno napisan.}
\item Da li rad sadrži sve elemente propisane uslovom seminarskog rada (slike, tabele, broj strana...)?\\ 
\odgovor {Da ispunjava sve uslove.}
\item Da li su slike i tabele funkcionalne i adekvatne?\\
\odgovor {Jesu.}
\end{enumerate}

\section{Ocenite sebe}
% Napišite koliko ste upućeni u oblast koju recenzirate: 
% a) ekspert u datoj oblasti
% b) veoma upućeni u oblast
% c) srednje upućeni
% d) malo upućeni 
% e) skoro neupućeni
% f) potpuno neupućeni
% Obrazložite svoju odluku
\odgovor {Malo sam upućen u jezik Swift. Do sada samo čuo za njega, nikad ga nisam koristio.}

\chapter{Recenzent \odgovor{--- ocena: 2} }


\section{O čemu rad govori?}
% Напишете један кратак пасус у којим ћете својим речима препричати суштину рада (и тиме показати да сте рад пажљиво прочитали и разумели). Обим од 200 до 400 карактера.
Rad govori o programskom jeziku Swift. O njegovom nastanku i razvoju kroz istoriju.
O njegovim osnovnim namenama kao što su razvoj aplikacija za iPhone i iPad uređaje.
Govori o podržanim paradigmam kao što su objektno-orjentisana, funkcionalna i imperativna.
Na kraju su objašnjena razvojna okruženja koja se koriste, instalacija programskog jezika Swift kao i osnovni primeri.

\section{Krupne primedbe i sugestije}
% Напишете своја запажања и конструктивне идеје шта у раду недостаје и шта би требало да се промени-измени-дода-одузме да би рад био квалитетнији.
U tekstu nije objašnjeno zašto je Swift programski jezik budućnosti kako stoji u naslovu teme.
Šta je to što ga razlikuje od ostalih programskih jezika.
Zašto bi čitaoce trebalo da zainteresuje programski jezik. 

\odgovor{Ispravljeno tako da naslov odgovara temi rada.}

\section{Sitne primedbe}
% Напишете своја запажања на тему штампарских-стилских-језичких грешки

Naslove tabela pisati bez tačke na kraju.
Izbegavati vertikalne linije u tabelama.

\odgovor{Obrisane su tačke kod naslova tabela.}

Slika 2 ispada iz leve margine. Trebalo bi da se pomeri udesno.

\odgovor{Slika je sada pomerena tako da ne ispada iz leve margine.}

\section{Provera sadržajnosti i forme seminarskog rada}
% Oдговорите на следећа питања --- уз сваки одговор дати и образложење

\begin{enumerate}
\item Da li rad dobro odgovara na zadatu temu?\\
Ne u potpunosti. U naslovu teme se napominje kako je Swift jezik budućnosti, ali to nije opisano u tekstu.

\odgovor{Promenjen je naslov tako da sada odgovara temi seminarskog rada.}

\item Da li je nešto važno propušteno?\\
Ne, seminarski rad sadrži adekvatnu strukturu. Autori su se potrudili da predstave jezik na najbolji mogući način.

\item Da li ima suštinskih grešaka i propusta?\\
Rad nema suštinskih grešaka.

\item Da li je naslov rada dobro izabran?\\
Smatram da naslov rada nije dobro izabran. Rad se više fokusira na upoznavanju čitalaca sa programskim jezikom,
nego šta to čini Swift programskim jezikom budućnosti.

\odgovor{Naslov je sada promenjen u \textbf{Razvoj i primena programskog jezika SWIFT}.}

\item Da li sažetak sadrži prave podatke o radu?\\
Sažetak sadrži prave podatke koji su opisani u radu.

\item Da li je rad lak-težak za čitanje?\\
Rad je lak za čitanje. Ne postoje pojmovi koji bi mogli biti nepoznati čitaocima a da nisu objašnjeni.

\item Da li je za razumevanje teksta potrebno predznanje i u kolikoj meri?\\
Za razumevanje teksta nije potrebno predznanje iz programskog jezika Swift. Potrebno je predznanje osnova programiranja.

\item Da li je u radu navedena odgovarajuća literatura?\\
Da, u radu je navedena odgovarajuća literatura.

\item Da li su u radu reference korektno navedene?\\
Da, sve reference u korektno navedene u tekstu.

\item Da li je struktura rada adekvatna?\\
Struktura rada je adekvatna. Rad sadrži uvod, razradu, zaključak i čitaoci se postepeno uvode u temu.

\item Da li rad sadrži sve elemente propisane uslovom seminarskog rada (slike, tabele, broj strana...)?\\
Rad sadrži slike, dijagram, tabele i ispunjeni su ostali uslovi seminarskog rada.

\item Da li su slike i tabele funkcionalne i adekvatne?\\
Da, slike i tabele su funkcionalne i adekvatne.

\end{enumerate}

\section{Ocenite sebe}
% Napišite koliko ste upućeni u oblast koju recenzirate: 
% a) ekspert u datoj oblasti
% b) veoma upućeni u oblast
% c) srednje upućeni
% d) malo upućeni 
% e) skoro neupućeni
% f) potpuno neupućeni
% Obrazložite svoju odluku
Potpuno sam neupućen. Nikada se nisam bavio razvojem aplikacija u programskom jeziku Swift, a ni razvojem mobilnih aplikacija generalno.


\chapter{Dodatne izmene}
%Ovde navedite ukoliko ima izmena koje ste uradili a koje vam recenzenti nisu tražili. 

\end{document}
