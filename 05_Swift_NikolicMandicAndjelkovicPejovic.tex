% !TEX encoding = UTF-8 Unicode
\documentclass[a4paper]{article}

\usepackage{color}
\usepackage{url}
\usepackage[T2A]{fontenc} % enable Cyrillic fonts
\usepackage[utf8]{inputenc} % make weird characters work
\usepackage{graphicx}
\usepackage{tabularx}
\newcolumntype{L}{>{\raggedright\arraybackslash}X}


\usepackage[english,serbian]{babel}
%\usepackage[english,serbianc]{babel} %ukljuciti babel sa ovim opcijama, umesto gornjim, ukoliko se koristi cirilica

\usepackage[unicode]{hyperref}
\hypersetup{colorlinks,citecolor=green,filecolor=green,linkcolor=blue,urlcolor=blue}

\usepackage{listings}
\usepackage{pgfplots}
\pgfplotsset{compat=1.10}
\usetikzlibrary{calc} 

\definecolor{RYB1}{RGB}{128, 177, 211}
\definecolor{RYB2}{RGB}{251, 128, 114}
\definecolor{RYB3}{RGB}{253, 180, 98}

\lstdefinelanguage{Swift}{
  keywords={associatedtype, class, deinit, enum, extension, func, import, init, inout, internal, let, operator, private, protocol, public, static, struct, subscript, typealias, var, break, case, continue, default, defer, do, else, fallthrough, for, guard, if, in, repeat, return, switch, where, while, as, catch, dynamicType, false, is, nil, rethrows, super, self, Self, throw, throws, true, try, associativity, convenience, dynamic, didSet, final, get, infix, indirect, lazy, left, mutating, none, nonmutating, optional, override, postfix, precedence, prefix, Protocol, required, right, set, Type, unowned, weak, willSet},
  ndkeywords={class, export, boolean, throw, implements, import, this},
  sensitive=false,
  comment=[l]{//},
  morecomment=[s]{/*}{*/},
  morestring=[b]',
  morestring=[b]"
}

\newcommand\todos[1]{\textcolor{red}{#1}}

\newtheorem{primer}{Primer}[section]

\definecolor{mygreen}{rgb}{0,0.6,0}
\definecolor{mygray}{rgb}{0.5,0.5,0.5}
\definecolor{mymauve}{rgb}{0.58,0,0.82}

\lstset{ 
  backgroundcolor=\color{white},   % choose the background color; you must add \usepackage{color} or \usepackage{xcolor}; should come as last argument
  basicstyle=\footnotesize,        % the size of the fonts that are used for the code
  breakatwhitespace=false,         % sets if automatic breaks should only happen at whitespace
  breaklines=true,                 % sets automatic line breaking
  captionpos=b,                    % sets the caption-position to bottom
  commentstyle=\color{blue},    % comment style
  deletekeywords={...},            % if you want to delete keywords from the given language
  escapeinside={\%*}{*)},          % if you want to add LaTeX within your code
  extendedchars=true,              % lets you use non-ASCII characters; for 8-bits encodings only, does not work with UTF-8
  firstnumber=1000,                % start line enumeration with line 1000
  frame=single,	                   % adds a frame around the code
  keepspaces=true,                 % keeps spaces in text, useful for keeping indentation of code (possibly needs columns=flexible)
  keywordstyle=\color{mygreen},       % keyword style
  language=Swift,                 % the language of the code
  morekeywords={*,...},            % if you want to add more keywords to the set
  numbers=left,                    % where to put the line-numbers; possible values are (none, left, right)
  numbersep=5pt,                   % how far the line-numbers are from the code
  numberstyle=\tiny\color{mygray}, % the style that is used for the line-numbers
  rulecolor=\color{black},         % if not set, the frame-color may be changed on line-breaks within not-black text (e.g. comments (green here))
  showspaces=false,                % show spaces everywhere adding particular underscores; it overrides 'showstringspaces'
  showstringspaces=false,          % underline spaces within strings only
  showtabs=false,                  % show tabs within strings adding particular underscores
  stepnumber=2,                    % the step between two line-numbers. If it's 1, each line will be numbered
  stringstyle=\color{mymauve},     % string literal style
  tabsize=2,	                   % sets default tabsize to 2 spaces
  title=\lstname                   % show the filename of files included with \lstinputlisting; also try caption instead of title
}

\begin{document}

\title{Razvoj i primena programskog jezika SWIFT \\ \small{Seminarski rad u okviru kursa\\Metodologija stručnog i naučnog rada\\ Matematički fakultet}}

\author{Anđelković Dragica, Mandić Igor, Nikolić Igor, Pejović  Petar\\ andjelkovic.dragica96@gmail.com,  igormandic996@gmail.com, \\ igor.nikolic032@hotmail.com, petar.pejovic8@gmail.com}

%\date{9.~april 2015.}

\maketitle

\abstract{
Swift je opšte primenljiv programski jezik za pisanje softvera, bilo da je to za mobilne telefone, desktop računare, servere ili bilo šta što pokreće k\^{o}d. To je bezbedan, brz i dinamičan programski jezik koji kombinuje najbolje u jednom savremenom jeziku. Swift spaja najbolja znanja iz široke Apple inženjering kulture i različite doprinose iz zajednice otvorenog k\^{o}da (eng.~{\em open source}). }


\tableofcontents

\newpage

\section{Uvod}
\label{sec:prviDeo}
Swift je relativno novi programski jezik opšte namene razvijen od strane kompanije Apple za iOS, macOS, watchOS, tvOS, Linux i z/OS. Dizajniran je da radi u Apple radnim okruženjima Cocoa i Cocoa Touch. Podržava imperativni, objektno orijentisani i funkcionalni način programiranja. Swift koristi izvršno okruženje programskog jezika Objective-C, što omogućava izvršavanje C, C++, Objective-C i Swift k\^{o}da u okviru jednog programa \cite{arc_sajt}. Namera kompanije Apple bila je da Swift podrži mnoge ključne koncepte povezane sa programskim jezikom Objective-C.

Cilj seminarskog rada jeste da se čitalac upozna sa osnovnim osobinama i funkcionalnostima programskog jezika Swift. Drugo poglavlje biće posvećeno istoriji nastanka programskog jezika Swift, kao i njegovom razvoju od prve verzije pa sve do danas. U trećem poglavlju biće opisane glavne mogućnosti i namena, dok je u četvrtom poglavlju dat pregled osnovnih osobina. Peto poglavlje će biti posvećeno razvojnim okruženjima, biće reči o osnovnim karakteristikama  razvojnih okruženja koje koristi Swift. U šestom poglavlju biće opisan način instalacije i pokretanja Swift-a na Windows i Linux operativnim sistemima. Primeri i kratka objašnjenja k\^{o}da biće data u sedmom poglavlju. Tema poslednjeg poglavlja će biti specifičnosti ovog programskog jezika.

\section{Nastanak i istorijski razvoj}
\label{sec:drugiDeo}
Chris Lattner je 2010. godine pokrenuo razvoj programskog jezika Swift. Swift je bio u tajnosti do kraja 2011. godine, kada su se i drugi programeri uključili u razvoj ovog jezika, a u julu 2013. godine postao je glavni projekat grupe Apple Developer Tools \cite{mastering_swift3}. 

Swift je predstavljen na međunarodnoj konferenciji programera (eng.~{\em Worldwide Developers Conference - WWDC}) 2014. godine, uz integrisano razvojno okruženje Xcode 6 \cite{thenextweb_sajt}. Apple je zvanično objavio Swift u decembru 2015. godine, kao projekat otvorenog k\^{o}da i pokrenuo je veb sajt \url{http://swift.org}. Na GitHub stranici kompanije Apple nalazi se repozitorijum (\url{https://github.com/apple/swift-evolution}) posvećen napretku Swift-a, tako da i drugi programeri mogu da budu deo procesa razvoja i poboljšanja Swift-a. U tabeli \ref{tab:razvoj} se nalaze sve do sada izbačene verzije programskog jezika Swift, u hronološkom redosledu.

\begin{table}[h!]
\begin{center}
\caption{Istorijski razvoj programskog jezika Swift}
\begin{tabular}{|c|c|} \hline
\label{tab:razvoj}
\textbf{Datum} & \textbf{Verzija} \\ \hline
2014-09-09 & Swift 1.0 \\ \hline
2014-10-22 & Swift 1.1 \\ \hline
2015-04-08 & Swift 1.2 \\ \hline
2015-09-21 & Swift 2.0 \\ \hline
2016-09-13 & Swift 3.0 \\ \hline
2017-09-19 & Swift 4.0 \\ \hline
2018-03-29 & Swift 4.1 \\ \hline
2018-09-17 & Swift 4.2 \\ \hline
2019-02-28 & Swift 4.3 \\ \hline
2019-03-25 & Swift 5.0 \\ \hline
\end{tabular}
\end{center}
\end{table}

Prvu verziju karakteriše REPL alat koji omogućava izvršavanje  manjih fragmenata Swift k\^{o}da i njegovo testiranje sa komandne linije \cite{swiftdev_sajt}. Ova verzija je donela poboljšanja performansi kompajlera i smanjila vreme potrebno za kompajliranje Swift programa. Prilikom pokretanja projekta, kompajliraju se samo fajlovi kod kojih je detektovana izmena, što je posebno značajno kod većih projekata. 

U drugoj verziji, kao deo novog projekta, predstavljen je Swift paket menadžer (eng.~{\em packet manager}) za upravljanje Swift bibliotekama. Kao priprema za naredne verzije dodata je provera verzije izvršnog okruženja (eng.~{\em framework}). 

Treća verzija sadrži osnovne promene u samom jeziku i biblioteci Swift standarda, zbog toga nije kompatibilna sa prethodnim verzijama jezika. Glavna ideja je bila da Swift bude kompatibilan na više platformi, što znači da će k\^{o}d koji se napše za macOS funkcionisati i na Linux-u. Tranzicija od druge verzije je bila veoma velika i teška programerima za ispravljanje, projekti su prijavljivali gomilu grešaka tako da su mnogi počinjani ispočetka. 

Nakon treće verzije, jezgro programskog jezika Swift nije se drastično menjalo, stoga su četvrta i peta verzija kompatibilne sa trećom. U narednim verzijama radilo se na poboljšanjima implementacije pojedinih koncepata jezika. Poslednja verzija koja je izbačena je verzija 5.0.
  
\subsection{Mesto u razvojnom stablu i uticaji drugih programskih jezika}
\label{subsec:podnaslovRazvojnoStablo}
Na razvoj Swift-a uticali su mnogi programski jezici, od kojih su najznačajniji: Objective-C, Rust, Haskell, Ruby, Python, C\#, CLU \cite{chris_sajt}. Mesto programskog jezika Swift u razvojnom stablu je predstavljeno na slici \ref{fig:razvojno_stablo}.

\begin{figure}[h!]
\begin{center}
\includegraphics[scale=0.5]{razvojno_stablo.png}
\end{center}
\caption{Razvojno stablo}
\label{fig:razvojno_stablo}
\end{figure}

Swift je preuzeo koncepte i rešenja, kako funkcionalnih, tako i objektno orijentisanih programskih jezika. Pored preuzimanja, vršena su i poboljšanja postojećih rešenja što je Swift dovelo u rang sa najzastupljenijim programskim jezicima. Pregled preuzetih koncepata se nalaze u tabeli \ref{tab:koncepti}.

\begin{table}[h!]
\begin{center}
\caption{Preuzeti koncepti iz drugih programskih jezika}
\begin{tabular}{|l|l|} \hline
\label{tab:koncepti}
\textbf{Programski jezik} & \textbf{Preuzeti koncepti} \\ \hline
JavaScript & Struktura podataka - rečnik  \\ \hline
Scala i Opa & Zaključivanje tipova \\ \hline
Cold Fusion i JSP & Interpolacija Stringa \\ \hline
Python & Opciono naznačavanje kraja naredbe \\ \hline
Java i C\# & Protokoli (Interfejsi) \\ \hline
Lisp i Python & Torke (eng.~{\em Tuples}) \\ \hline
Lisp i JavaScript &  Closure funkcije \\ \hline
C\# i Objective-C & Označeni i neoznačeni celi brojevi \\ \hline
\end{tabular}
\end{center}
\end{table}

\section{Osnovna namena, svrha i mogućnosti}	
\label{sec:treciDeo}
Pomoću Swift programskog jezika moguće je razviti bilo koji tip iOS i macOS aplikacija. Cilj Swift projekta je da stvori najbolji raspoloživi jezik za upotrebu, od programiranja sistema, preko razvoja mobilnih i desktop aplikacija do cloud usluga. Takođe, ubrzan je proces razvoja proizvoda, poboljšane su performanse i povećana sigurnost aplikacija.

Jedna od glavnih namena Swift programskog jezika je kreiranje mobilnih aplikacija za iPhone i iPad uređaje. Pored macOS, Swift je moguće izvršavati i na Linux operativnom sistemu. Članovi zajednice rade na stvaranju Swift aplikacija koje će se izvršavati i na Android platformama.

Osim što je Swift poznat po razvoju aplikacija za Apple platforme, koristi se i u modernim server aplikacijama. Swift je odličan izbor za server aplikacije koje zahtevaju visoke performanse kompajlera, nizak stepen korišćenja memorije i visok nivo bezbednosti.

Danas je Swift sve popularniji programski jezik za razvoj IoT (eng.~{\em Internet of Things}) aplikacija. Kako bi kompanija Apple postala lider u primeni IoT aplikacija, razvijene su biblioteke i razvojni okviri koje rade najveći deo posla, dok se programeri mogu fokusirati na funcionalnosti IoT aplikacija.

Swift ima svoju standardnu biblioteku. Biblioteka sadrži osnovne funkcionalnosti za pisanje Swift programa, uključujući tipove podataka, strukture podataka, funkcije, metode, protokole, i mnogo drugih stvari \cite{naucni_rad}.

Swift pruža brojne mogućnosti koje omogućavaju brže i lakše pisanje k\^{o}da. Tako na primer pomoću mehanizma automatskog utvrđivanja tipova Swift prepoznaje tip promenljive na osnovu njene početne vrednosti. Koristeći sintaksu zatvoreog izraza Swift poštuje princip ponovne upotrebe k\^{o}da. Primena generičkih tipova smanjuje potrebu za pisanjem duplog k\^{o}da za različite tipove podataka. Detaljniji opis Swift funkcija se nalaze u dodatku \ref{sec:dodatak}.


\section{Osnovne osobine i podržane paradigme}	
\label{sec:cetvrtiDeo}
Swift je objektno orijentisan programski jezik, koji je Apple razvio sa ciljem da se poboljšaju određeni delovi jezika Objective-C, ali da se i iskoriste njegove dobre osobine. Najvažnije osobine programskog jezika Swift, koje ga čine izuzetnim za učenje iOS programiranja su \cite{swift_programming}:

\begin{itemize}
\item\textbf{Objektno orijentisan} - sadrži osnovne koncepte objektno orijentisanih programskih jezika
\item\textbf{Funkcionalan} - sadrži osobine zbog kojih je pogodan za pisanje funkcionalnih programa
\item\textbf{Jasan} - lako se čita i piše, njegova sintaksa je jasna, dosledna i očigledna
\item\textbf{Bezbedan} - zahteva jake tipove, u svakom trenutku i jezik i programer znaju na šta se sve tipovi objekata pozivaju
\item\textbf{Ekonomičan} - mali jezik koji nudi samo neke osnovne tipove podataka i funkcionalnosti, preostalo mora da bude dato kodom programera ili bibliotekama
\item\textbf{Upravlja memorijom} - automatski upravlja memorijom, programer ne treba o tome da brine
\item\textbf{Kompatibilnost sa razvojnim okruženjem Cocoa} -  Swift je napravljen tako da koristi većinu API-ja (Application Programming Interface) razvojnog okruženja Cocoa
\end{itemize}

Zbog svih ovih osobina, 2016. godine, programski jezik Swift je bio najplaćeniji jezik u Americi. Na sledećem grafikonu (slika \ref{fig:grafikon}) prikazano je 10 najplaćenijih programskih jezika:

\begin{figure}[hbt!]
\hspace{2.5em}
\begin{tikzpicture}
\pgfplotsset{width=10 cm, height = 5cm}
\begin{axis} [
symbolic x coords={Swift, Python, Ruby, C++, Java, JavaScript, C, SQL, PHP},
xtick={Swift, Python, Ruby, C++, Java, JavaScript, C, SQL, PHP},
x tick label style={rotate=45, anchor=east, align=center},
axis lines*=left,
ymajorgrids = true,
ylabel=Zarada u hiljadama,
legend style={at={(0.5,-0.10)},
    anchor=north,legend columns=1},
    ymin=80,
    ytick={80,90,...,120},
    ymax=120,
    bar width=5mm,
    ybar=-0.5cm, 
   enlarge x limits={abs=0.6cm},
    nodes near coords,        
    every node near coord/.append style={color=black},
]
\addplot [RYB3,fill=RYB3]
coordinates{ (Swift,115) } ;
\addplot [RYB2,fill=RYB2]
coordinates{ (Python,107) } ;
\addplot [RYB1,fill=RYB1]
coordinates{ (Ruby,107)  } ;
\addplot [RYB2,fill=RYB2]
coordinates{ (C++,104) } ;
\addplot [RYB2,fill=RYB2]
coordinates{ (Java,102) } ;
\addplot [RYB1,fill=RYB1]
coordinates{ (JavaScript,99) } ;
\addplot [RYB3,fill=RYB3]
coordinates{ (C,94)  } ;
\addplot [RYB1,fill=RYB1]
coordinates{ (SQL,92) } ;
\addplot [RYB3,fill=RYB3]
coordinates{ (PHP,89)  } ;

\end{axis}

\end{tikzpicture}
\caption{Najplaćeniji programski jezici u Americi 2016. godine}
\label{fig:grafikon}
\end{figure}


Programski jezik Swift je \textbf{objektno orijentisani} jezik, ali podržava i \textbf{funkcionalno} i \textbf{imperativno}
programiranje. Sadrži sve osnovne koncepte objektno orijentisanog programiranja, i objektno orijentisana paradigma je najzastupljenija u ovom jeziku. Međutim, nekoliko osobina Swift jezika, kao što su funkcije prvog reda, sofisticirani sistem tipizacije, lambda izrazi, korišćenje Karijevih funkcija i parcijalna aplikacija, čine jezik posebno pogodnim za pisanje funkcionalnih programa \cite{rad}. Swift, kao funkcionalni jezik, može se u svakodnevnoj praksi koristiti za pisanje kraćih, elegantnijih i sigurnijih programa, koji su lakši za održavanje, nadogradnju i testiranje.

\section{Okruženja i njihove karakteristike}	
\label{sec:petiDeo}
Programski jezik Swift se može pisati u različitim okruženjima. Najpoznatije okruženje je \textbf{Xcode}, a pored njega, koriste se i Playground, Cocoa Touch, Atom, SublimeText.

\subsection{Xcode}
\label{subsec:podnaslovXcode}
Xcode  je integrisano razvojno okruženje koje je napravila kompanija Apple. Koristi se za razvoj iOS i macOS aplikacija. U ovom okruženju mogu se pisati kodovi u mnogim programskim jezicima, kao što su C, C++, Objective-C, Java, AppleScript, Python, Ruby i Swift. Xcode sadrži i okvire, za programski jezik Swift su najvažniji \textbf{Playground} i \textbf{Cocoa Touch}. 
Postoji 9 verzija ovog okruženja, a počev od verzije 6, obuhvata se programski jezik Swift.

\subsection{Playground}
\label{subsec:podnaslovPlayground}
Playground je interaktivno radno okruženje koje omogućava pisanje kodova, a rezultati se vide odmah, čim su promene izvršene u k\^{o}du. Da bi se koristio ovaj okvir potrebno je prvo pokrenuti Xcode i zatim izabrati opciju Get started with a playground. Okruženje se sastoji od nekoliko delova, među kojima su najvažniji \cite{mastering_swift3}:
 
\begin{itemize}
\item Prostor za kodiranje
\item Bočna traka za rezultate
\item Prostor za ispravljanje grešaka
\end{itemize}

\subsection{Cocoa Touch}
\label{subsec:podnaslovCocoaTouch}
Cocoa Touch je okruženje prvenstveno napravljeno za razvoj programa
koji su namenjeni uređajima koji koriste iOS. Ovo okruženje je napisano u Objective-C-u. Omogućava upotrebu hardvera i karakateristika koje nisu implementirani na macOS računarima, već je jedinstvena za iOS asortiman uredjaja. Sadrži različite skupove alata za kontrolu grafičkih elemenata. Alati za razvoj aplikacija ovog okruženja uključeni su u iOS SDK (Software Development Kit).

\subsection{Sublime Text i Atom}
\label{subsec:podnaslovAtom}
Sublime Text i Atom, iako nisu prvenstveno namenjeni za razvijanje Swift programa, verovatno su jedni od najrasprostranjenijih i najpoželjnijih okruženja za rad. Poseduju dobar korisnički interfejs i sjajne performanse. U osnovi podržavaju mnoge jezike, a nove funkcionalnosti mogu biti dodate korišćenjem dodataka (eng.~{\em plugina}). Vrlo lako ih možemo prilagoditi za razvoj Swift aplikacija dodatkom podrške za Swift pakete.

\section{Instalacija i uputstvo za pokretanje}	
\label{sec:sestiDeo}
Programski jezik Swift se može koristiti na različitim operativnim sistemima. Ukoliko se koristi macOS, dovoljno je da se preuzme i instalira Xcode razvojno okruženje, jer Xcode uključuje izdanje Swift-a. Za Linux i Windows, instalacija i pokretanje su složeniji, i biće prikazani u ovom poglavlju.

\subsection{Windows}
\label{subsec:podnaslovWindows}
Za instalaciju programskog jezika Swift na operativnom sistemu Windows potrebno ga je prvo preuzeti sa ovog \href{https://swiftforwindows.github.io}{linka}. Nakon toga, pojavljuje se prozor za instalaciju, gde se prate dalja uputstva i tako instalira Swift i kompajler za Swift. Nakon završetka instalacije, potreban je editor teksta u kojem se piše k\^{o}d. U ovom primeru koristi se \href{https://notepad-plus-plus.org/download/v7.6.4.html}{Notepad++}, koji je jednostavan, besplatan i lak za instalaciju.

Nakon instalacije okruženja, treba napisati jednostavan program koji će se kasnije pokrenuti pomoću Windows komandne linije. Prvo je potrebno otvoriti novi Notepad++ fajl, i u njega uneti komandu za ispis.

\begin{lstlisting}[language=bash, caption={Primer komande}]
print("Hello world!")
\end{lstlisting}

Za čuvanje ovog k\^{o}da, koristi se komanda File > Save As i bira se Swift file iz Save As Type menija. Ako u meniju nedostaje tip ovog fajla, potrebno je izabrati all files, i dodati .swift fajl ekstenziju, nakon što je dodeljeno odgovarajuće ime fajlu.

Kada je program napisan i sačuvan, potrebno ga je kompajlirati i pokrenuti. Za kompajliranje i pokretanje koristi se korisnički interfejs Swift-a koji je prikazan na slici \ref{fig:windows}.

\begin{figure}[h!]
\begin{center}
\includegraphics[scale=0.27]{swift-win.png}
\end{center}
\caption{Korisničko okruženje za Windows}
\label{fig:windows}
\end{figure}

Nakon pritiska na Select File, potrebno je izabrati program i pritisnuti Compile. Nakon što se kompilacija završi dobija se poruka “Successfully compiled”.
Jednom kompajliran program može se pokrenuti neograničen broj puta.

\subsection{Linux}
\label{subsec:podnaslovLinux}
Kao i u prethodnom primeru, potreban je tekst editor gde će se napisati jednostavan k\^{o}d.
Može se koristiti bilo koji integrisani editor koji Linux poseduje. Na isti način, kodira se poruka za ispis "Hello World", kao u prethodnom primeru, i taj fajl se čuva sa ekstenzijom .swift.

Da bi se koristio Swift na Linux operativnom sistemu mora se prvo instalirati. U terminalu se kucaju sledeće komande:

\begin{lstlisting}[language=bash, caption={Instaliranje Swift-a}]
	wget https://swift.org/builds/ubuntu1510/swift-2.2-SNAPSHOT-2015-12-10-a/swift-2.2-SNAPSHOT-2015-12-10-a-ubuntu15.10.tar.gz
\end{lstlisting}

Nakon preuzimanja, potrebno je pozicionirati se u folder Downloads i tamo raspakovati arhivu u kojoj se nalazi Swift instalacija.

\begin{lstlisting}[language=bash, caption={Raspakovanje Swift instalacije}]
  cd ~/Downloads
  tar -xvzf swift-2.2-SNAPSHOT*
\end{lstlisting}

Za raspakivanje fajla potrebno je podesiti putanju do BIN-a, kako bi programi mogli da se izvršavaju.

\begin{lstlisting}[language=bash, caption={Podešavanje putanje do BIN-a}]
  cd ~/Downloads/swift-2.2-SNAPSHOT*
  cd usr/bin
  pwd
\end{lstlisting}

Kao rezultat komande pwd dobija se tačna lokacija koja će se koristiti. Ona se kopira i zameni na sledeći način:
\begin{lstlisting}[language=bash, caption={Kopiranje putanje}]
  export PATH=path_to_swift_usr_bin:$PATH
\end{lstlisting}

Zatim je potrebno instalirati jos par biblioteka kako bi omogućili da Swift nesmetano funkcioniše na Linux-u.

\begin{lstlisting}[language=bash, caption={Instalacija biblioteka}]
  sudo apt-get install clang libicu-dev
  swift -version
\end{lstlisting}

Na kraju, za kompajliranje i pokretanje prethodno napisanog programa, potrebno je ukucati sledeće komande:

\begin{lstlisting}[language=bash, caption={Komanda za kompajliranje}]
  swift imeprograma.swift
  ./imeprograma
\end{lstlisting}
 
\section{Primer jednostavnog koda i njegovo objašnjenje}	
\label{sec:sedmiDeo}
U narednim primerima biće prikazane i objašnjene osnovne funkcionalnosti jezika Swift.

\begin{primer}
Ispis teksta se vrši pomoću funkcije print(). Tačka-zarez su opcioni na kraju svakog reda.
\end{primer}

\begin{lstlisting}[language=Swift, caption={Ispis teksta},frame=single, label=simple]
print("Hello world!")				    // Hello World!
print("Hello world!");					// Hello World!
\end{lstlisting}

\begin{primer}
Stringovi se takođe dodeljuju pomoću operatora dodele. Konkatenacija Stringova se vrši pomoću specijalnih karaktera '$\backslash$ (string)' ili jednostavnim navođenjem u naredbi 'print', gde se Stringovi razdvajaju zarezima.
\end{primer}

\begin{lstlisting}[language=Swift, caption={Stringovi i konkatenacija stringova},frame=single, label=simple]
var ime = "Swift"
var jezik = "programski jezik"

var poruka = " je najbolji "
var poruka1 = "\ (ime) je najbolji \ (jezik) !" 

print(ime,poruka,jezik,"!")
// Swift je najbolji progmski jezik!
print(poruka1) 
// Swift je najbolji progmski jezik!
\end{lstlisting}

\begin{primer}
Lista Stringova koja je razdvojena određenim separatorom pravi se pomoću naredbe print, gde se prvo navode Stringovi koji čine tu listu, a nakon toga separator i terminator. Može se koristiti još jedan parametar u funkciji print(), pod nazivom toStream. Pomoću njega preusmerava se ispis funkcije print(). Konkretno u ovom primeru preusmerava se ispis u promenljivu ime4.
\end{primer}

\begin{lstlisting}[language=Swift, caption={Lista stringova},frame=single, label=simple]
var ime1 = "Swift"
var ime2 = "Java"
var ime3 = "Python"
var ime4 = ""

print(ime1,ime2,ime3,separator:", ", terminator:"") 
// Swift, Java, Python
print(ime1,ime2,ime3,separator:", ", terminator:"", to:&ime4)
// Swift, Java, Python
\end{lstlisting}

\begin{primer}
U ovom primeru je pokazana funkcionalnost for i while petlje. Takođe, još jednom i mogućnost konkatenacije pomoću operatora +. Nakon for petlje u konzoli će biti ispisani brojevi od 0 do 10. U while petlji će se svakog puta dodavati po jedno slovo a, i tako 5 puta.
\end{primer}

\begin{lstlisting}[language=Swift, caption={Petlje},frame=single, label=simple]
var x:Int = 0
for x in 0...10 {
	print(x)
}

var y:String = ""
while y != "aaaaa" {
	print(y)
	y = y + "a"
}
\end{lstlisting}

\begin{primer}
Pozivanje funkcije sa parametrima koja nema povratnu vrednost. Sintaksa za deklaraciju funkcija se sastoji iz ključne reči func, naziva i liste parametara. Parametri su oblika naziv ':' tip promenljive.
\end{primer}

\begin{lstlisting}[language=Swift, caption={Funkcije bez povratne vrednosti},frame=single, label=simple]
var skor:Int = 0
var trenutno_stanje:Float = 0

func dodaj_poene_i_novac(poeni:Int, novac:Float){
	skor = skor + poeni
	trenutno_stanje = trenutno_stanje + novac
}

dodaj_poene_i_novac(poeni: 30, novac: 1.45)
dodaj_poene_i_novac(poeni: 60, novac: 2.86)

print(skor)								// 90
print(trenutno_stanje)					// 4.31
\end{lstlisting}

\begin{primer}
Funkcije u programskom jeziku Swift dozvoljavaju vraćanje jedne ili više vrednosti istovremeno. Sintaksa za povratnu vrednost je 'return(niz vrednosti koje se vraćaju)'.
\end{primer}

\begin{lstlisting}[language=Swift, caption={Funkcije koje imaju povratnu vrednost},frame=single, label=simple]
func izracunajMinMaxSuma(a: Int, b: Int) -> (min: Int, max: Int, suma: Int) {
    
    if a > b {
        return (b, a, a + b)
    } else {
        return (a, b, a + b)
    }
}
let statistika = izracunajMinMaxSuma(5, b: 19)
let (min2, max2, suma2) = izracunajMinMaxSuma(5, b: 19)
print(suma2)            						// 24
print(statistika.sum)   						// 24
print(statistika.2)								// 5     
\end{lstlisting}

\begin{primer}
Klase u programskom jeziku Swift se definišu sa 'class', nakon čega sledi ime klase. Klase sadrže polja, konstruktor i metode. Poljima klase se pristupa pomoću 'self', dok se konstruktor definiše pomoću ključne reči 'init',
\end{primer}

\begin{lstlisting}[language=Swift, caption={Klasa},frame=single, label=simple]
class Osoba {
    var ime: String
    var godine: Int
    
    init(ime: String, godine: Int) {
        self.ime = ime
        self.godine = godine
    }
    func getIme() -> String {
        return "Tvoje ime je \(self.ime)"
	}
}
var osoba1 = Osoba(ime: "Daca", godine: 22)
print(osoba1.getIme))								// Daca
\end{lstlisting}

\section{Specifičnosti}	
\label{sec:osmiDeo}
Swift je pristupačan novim programerima, to je industrijski kvalitetan programski jezik koji je veoma detaljan i pogodan kao skriptni jezik. Swift se dobro štiti od najzastupljenih programskih grešaka usvajanjem modernih paterna programiranja:
\begin{itemize}
\item Promenljive su uvek inicijalizovane pre upotrebe
\item Obrađena je greška za pristupanje nepostojećem elementu niza (eng.~{\em out of bounds})
\item Celi brojevi (eng.~{\em integers}) su provereni za prekoračenje memorije (eng.~{\em overflow})
\item Opcione promenljive zahtevaju eksplicitno rukovanje
\item Memorijom se upravlja automatski
\item Rukovanje greškama omogućava kontrolisani oporavak od neočekivanih prekida (eng.~{\em crash})
\end{itemize}

Swift kombinuje paterne sa modernom laganom sintaksom omogućavajući da složene ideje budu predstavljene na koncizan način, i kao rezultat k\^{o}d je lakši za pisanje, čitanje i održavanje. Swift k\^{o}d je kompajliran i optimizovan da izvuče najviše iz modernog hardvera. Ova kombinacija sigurnosti i brzine čini Swift odličnim izborom za sve, od komande "Hello world!", do celog operativnog sistema.

\section{Zaključak}
\label{sec:zakljucak}
Swift je mlad programski jezik, koji iz godine u godinu napreduje i postaje popularniji u iOS programiranju. Očekuje se njegova ekspanzija u narednom periodu zbog sve većeg korišćenja iOS aplikacija. Naime, godinama se radilo na Swift-u i on nastavlja da evoluira sa novim funkcionalnostima, a cilj koji ima je veoma ambiciozan. Stoga, naš zaključak jeste da Swift programski jezik predstavlja budućnost obzirom da postoje velike mogućnosti za njegovu primenu i dalje usavršavanje.

\addcontentsline{toc}{section}{Literatura}
\appendix
\bibliography{seminarski} 
\bibliographystyle{plain}


\appendix

\section{Dodatak}
\label{sec:dodatak}

Neke od Swift funkcija se nalaze u tabeli \ref{tab:funkcije} koja je preuzeta iz knjige Naučite Swift 3, čiji je autor Jon Hoffman \cite{mastering_swift3}. Pored funkcija koje se nalaze u tabeli, postoji još jedna bitna funkcija za programski jezik  Swift, a to je funkcija Xcode-a i kompajlera - \textbf{Mix and match}, koja  omogućava kreiranje aplikacija koje sadrže Objective-C i Swift fajlove.

\begin{table}[h!]
\centering
\caption{Pregled Swift funkcija}
\label{tab:funkcije}
\begin{tabular}{|p{3cm}|p{7cm}|} 

\hline
\textbf{Swift funkcija} & \textbf{Opis} \\ \hline
automatsko utvrđivanje   \hspace{10mm} tipova & Swift može automatski da utvrdi tip promenljive ili konstante na osnovu inicijalne
vrednosti. \\ \hline
generički tipovi & Generički tipovi omogućavaju da se piše kod jednom za izvršenje identičnih
zadataka za različite tipove objekata dok se zadržava bezbednost tipa. \\ \hline
promenljivost  kolekcije & Swift nema posebne objekte za promenljive ili nepromenljive kontejnere. Umesto
toga, možete da definišete promenljivost definisanjem kontejnera kao konstante
ili kao promenljive. \\ \hline
sintaksa zatvorenog izraza & Zatvoreni izrazi su samostalni blokovi funkcionalnosti koji mogu da se proslede i
upotrebe u kodu. \\ \hline
pseudoklase & Pseudoklasa definiše promenljivu koja možda nema vrednost. \\ \hline
switch iskaz &  Switch iskaz je drastično poboljšan funkcijama, kao što su poklapanje šablona i
zaštitni uslovi; zahvaljujući njima, izbegnute su automatske greške. \\ \hline
višestruki povratni tipovi &  Funkcije mogu da imaju višestruke povratne tipove upotrebom torki. \\ \hline
preklapanje  \hspace{10mm} operatora & Klase mogu da obezbede sopstvenu implementaciju postojećih operatora. \\ \hline
nabrajanja sa pratećim  \hspace{10mm} vrednostima & U Swiftu može da se uradi mnogo više od jednostavnog definisanja grupe srodnih
vrednosti pomoću nabrajanja. \\ \hline 

\end{tabular}
\end{table}


\end{document}